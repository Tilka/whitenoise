\documentclass[12pt,oneside,notitlepage]{article}
\usepackage{times}
\usepackage[dvips]{graphics}
\usepackage{fullpage}
\usepackage{longtable}

% Set up the page headers and footers
\usepackage{fancyhdr}
\pagestyle{fancy}
\lhead{whitenoise 1.0}
\rhead{User Documentation}
\rfoot{Paul J. Pelzl}
\renewcommand{\headrulewidth}{0.2pt}
\renewcommand{\footrulewidth}{0.2pt}
\renewcommand{\headheight}{0.5in}

\renewcommand{\arraystretch}{2.0}


\title{whitenoise v1.0 \\ User Documentation}
\author{Copyright \copyright 2001, 2002, 2004, 2010 Paul Pelzl}
\date{September 25, 2010}


%%%%%%%%%%%%%%%%%%%%%%%%%%%%%%%%%%%%%%%%%%%%%%%%%%%%%%%%%%%%%%%%%%%%%%%%%%
\begin{document}

\maketitle

\section{Introduction}
{\tt whitenoise} is a little utility that allows your computer to act as an ambient
random noise generator.  You may find this useful for drowning out noisy
neighbors, helping you sleep, etc.


\section{Usage}

\subsection{Command-line syntax}
Syntax:  {\tt whitenoise [option(s)]}

\noindent
options:

%BEGIN LATEX
\begin{longtable}{lp{4in}}
%END LATEX
%HEVEA \begin{tabular}{lp{4in}}
   {\tt -c CUTOFF} &    Sets the lowpass frequency cutoff.
                        {\tt CUTOFF} is a number in the range {\tt (0, 1)},
                        with a default value of {\tt 0.3}. \\
   {\tt -r RATE} &      Sets the samplerate for the sound card, in Hz.
                        Acceptable values are {\tt 11025} and {\tt 22050}, with
                        a default value of {\tt 22050}. \\
   {\tt -F FILTNUM} &   Use a filter of type {\tt FILTNUM}, which may
                        take the following values:
                        \newcounter{filt}
                        \begin{list}{ {\tt \arabic{filt}} :}
                           {\usecounter{filt}
                           \setcounter{filt}{-1}}
                           \item Blackman-windowed FIR lowpass (default)
                           \item Bartlett-windowed FIR lowpass
                           \item Hanning-windowed FIR lowpass
                           \item Hamming-windowed FIR lowpass
                           \item Rectangular-windowed FIR lowpass
                        \end{list} \\
  {\tt -l LENGTH} &     Sets the FIR filter length.
                        {\tt LENGTH} is an integer in the range {\tt [1, 100]},
                        with a default value of {\tt 25}. \\
  {\tt -t TIME} &       Sets the length of time to generate
                        noise, in minutes. \\
  {\tt -f FADETIME} &   Fade the noise out over {\tt FADETIME}
                        seconds.  Valid only when used along with
                        the {\tt -t} flag. \\
  {\tt -p WIDTH} &      Output a plot of the filter frequency response,
                       {\tt WIDTH} is the horizontal resolution of the
                        PNG image, with default 320.  The image will
                        have filename {\tt \~{}/.whitenoise/filter.png} . \\
  {\tt -L LATENCY} &    Configure the audio buffers for approximately
                        {\tt LATENCY} milliseconds of delay, with default
                        200.  Increase the value to alleviate
                        problems with skipping. \\
  {\tt -a} &            Interface with aRts instead of opening
                        /dev/dsp directly. \\
  {\tt -s} &            Read commands from stdin in realtime. \\
  {\tt -v, --version} & Print version information. \\
  {\tt --help, -?} &    This help page. \\
%HEVEA \end{tabular}
%BEGIN LATEX
\end{longtable}
%END LATEX

Increasing or decreasing the frequency cutoff value will increase or decrease
the amount of high frequency content in the noise.  The samplerate can be
lowered to 11025 Hz for "warmer" noise that is dominated by lower frequencies.
Choosing a different filter will impact the overall balance of frequencies in
the noise.  Increasing the filter length will make the lowpass filter more
ideal, at the cost of increased CPU usage.

If whitenoise tends to skip (for example, under high CPU load), then it may
help to increase the latency via the ``{\tt -L}" option.

The ``{\tt -p}" option is available only if whitenoise is compiled with support
for FFTW 3.x.  Similarly, the ``{\tt -a}" option is only available when whitenoise
has been compiled against aRts.

When the ``{\tt -s}" option is used, whitenoise will continually read commands
from stdin.  This may be useful for creating a frontend to control
whitenoise ({\tt gnome-whitenoise} is one example).  See Section \ref{stdin} for
further information.

\subsection{Controlling whitenoise via standard input}
\label{stdin}
If {\tt whitenoise} is launched in interactive mode via the ``{\tt -s}'' option,
then it will continually scan standard input for command strings.
Acceptable command strings take the form
\begin{verbatim}
xaaaaaa...
\end{verbatim}
and should be terminated with newline.  `{\tt x}' represents a command character; the
possible characters are the same as the command-line switches: \{ {\tt c, r, F, l, t, f, p}\}.
``{\tt aaaaa....}" is a string providing the argument of the command.  The character `{\tt q}'
can also be used to terminate whitenoise.  Some special cases deserve attention:

\begin{itemize}
   \item Using the ``{\tt tTIME}" command will reset the timer; i.e. the command ``{\tt t30}''
     will shut off whitenoise 30 minutes after the command is entered.
  \item To cancel the timer or the fade option, use ``{\tt -1}'' as the argument to those
     commands.
\end{itemize}

You should expect a short delay between entering a command and hearing the
result.


\section{Requirements}
{\tt whitenoise} requires a sound card with Open Sound System-compatible
drivers.  I use GNU/Linux for development, but I expect that it should work
without difficulty on *BSD machines, and users have previously reported that
{\tt whitenoise} also works fine on Windows via 
%BEGIN LATEX
Cygwin\footnote{ {\tt http://www.cygwin.com/}}.
%END LATEX
%HEVEA \begin{rawhtml} <a href="http://www.cygwin.com">Cygwin</a>. \end{rawhtml}
On very low-end systems, it may
be necessary to decrease the filter length to prevent skipping.

The 
%BEGIN LATEX
aRts\footnote{ {\tt http://www.arts-project.org/} }
%END LATEX
%HEVEA \begin{rawhtml} <a href="http://www.arts-project.org/">aRts</a> \end{rawhtml}
library and header files 
are required only if you want whitenoise to support output to artsd, the sound
server that ships with KDE.  (Specifically, {\tt artsc.h} and libartsc are
required.)

%BEGIN LATEX
FFTW v3.x\footnote{ {\tt http://www.fftw.org/} }
%END LATEX
%HEVEA \begin{rawhtml} <a href="http://www.fftw.org/">FFTW v3.x</a> \end{rawhtml}
is required if you want whitenoise
to be able to generate plots of the filter frequency response. 
%BEGIN LATEX
gnuplot\footnote{ {\tt http://www.gnuplot.info/} }
%END LATEX
%HEVEA \begin{rawhtml} <a href="http://www.gnuplot.info/">gnuplot</a> \end{rawhtml}
is also a requirement for plotting, although it is not needed during compilation.



\section{Installation}
In general, enter the root of the {\tt whitenoise} source directory and use
\begin{verbatim}
$ ./configure
$ make
\end{verbatim}
to compile the executable (which is, of course, called {\tt whitenoise}).  Run
\begin{verbatim}
$ make install
\end{verbatim}
to install it (you may need to be root to have permission to copy the executable
to a system directory).

You can use
\begin{verbatim}
$ ./configure --help
\end{verbatim}
to see some of the options that {\tt configure} accepts.  In particular, you may be interested in
the options that identify the locations of the aRts and FFTW headers and libraries.  For example, on my
system I use 
\begin{verbatim}
$ ./configure --with-arts-inc=/usr/include/kde/artsc
\end{verbatim}
to properly detect {\tt artsc.h}.


\section{License}
{\tt whitenoise} is Free Software, released under the GNU General Public License.
You are free to modify and redistribute {\tt whitenoise} under certain conditions; 
see {\tt COPYING} for details.



\section{How it works}
Uniform random noise is generated with rand().  This noise sounds awful because
it has too much high-frequency content, so it is lowpass filtered.  There are
a number of standard filters available for this purpose, all of which are 
designed using the window method.  Yes, I am aware that the noise generated 
by this scheme is not technically ``white", but real white noise has rather 
disturbing audio characteristics.



\section{Future work}
It is unlikely that there will be any significant changes in the future.



\section{Contact info}
Feel free to contact me if you:
\begin{itemize}
   \item like whitenoise
   \item hate whitenoise
   \item find bugs
\end{itemize}

%BEGIN LATEX
\renewcommand{\arraystretch}{1.0}
\begin{tabular}{ll}
   {\tt whitenoise} homepage: & {\tt http://pessimization.com/software/whitenoise} \\
   my email: &                  {\tt <pelzlpj@gmail.com>}
\end{tabular}
%END LATEX
%HEVEA \begin{rawhtml} <a href="http://pessimization.com/software/whitenoise">whitenoise homepage</a><br> \end{rawhtml}
%HEVEA \begin{rawhtml} <a href="mailto:pelzlpj@gmail.com">send me email</a> \end{rawhtml}


\end{document}



% arch-tag: DO_NOT_CHANGE_8527ca15-717f-4d4d-81df-978ea00f6736 
